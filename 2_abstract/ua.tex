\chapter*{Реферат}

Дисертація містить 
%\pageref{LastPage} 
сторінки,
%\TotalValue{totalfigures} 
ілюстрацій
і бібліографію з 
%\total{citenum} 
найменувань.

Дану роботу присвячено алгоритму, який для будь-якої
поданої на вхід \((max,+)\) задачі розмітки з цілочисельними
якостями надасть одну з двох відповідей: або оптимальну
розмітку, або «задача не супермодулярна», і ця
відповідь гарантовано буде коректною.
Самоконтроль полягає у тому, що не користувач вирішує, 
на яке питання треба відповісти, а сам алгоритм вирішує, 
що потрапляє у зону його компетентності. Іншою
особливістю алгоритму є те, що він не потребує відомої 
впорядкованості міток для супермодулярних задач.

Гарантію скінченної кількості кроків надає використання
субградієнтного спуску і цілочисельність ваг вершин та ребер.

\MakeUppercase{\((max,+)\) задачі розмітки,
супермодулярні задачі розмітки,\\
самоконтроль у розпізнаванні образів, 
дискретна оптимізація, графові моделі,
структурне розпізнавання образів.}
