\chapterConclusion

В розділі було розглянуто основні методи розв’язування $(\max ,+)$ задач розмітки.
Було описано переваги та недоліки кожного з них. Алгоритми, які розв'язують
задачу точно, є більш бажаними, але часто вони вирішують задачі лише обмеженого
класу (задачі з двома мітками, супермодулярні задачі із відомою впорядкованістю
міток тощо). Ітеративні методи дозволяють розв'язувати значно ширший клас задач.
Основними недоліками таких алгоритмів є їх повільність та відсутність будь-яких
гарантій на відшукання точного розв'язку, а також проблема відшукання оптимальної
розмітки після досягнення прийнятного значення цільової функції.

Наведено постановку $(\max,+)$ задачі розмітки та показано перехід до двоїстої задачі.
Розглянуто властивість супермодулярності.
Ці дані є необхідними для конструювання алгоритму у наступному розділі дисертації.
