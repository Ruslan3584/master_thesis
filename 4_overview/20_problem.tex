\section{Постановка \((\max,+)\) задачі розмітки}

Задано скінченну непорожню множину $T$ об'єктів та скінченну
непорожню множину $K$ міток. Функцію $k:T\rightarrow K$ будемо
називати розміткою, що для кожного об'єкту $t\in T$ визначає мітку
$k(t)\in K$. На множині $t$ об'єктів визначимо структуру сусідства
$\Gamma \subset T^2$, яка є асиметричною
\begin{equation*}
    (t,t') \in \Gamma\implies (t',t) \notin \Gamma.
\end{equation*}
Надалі буде використовуватися запис $tt'$ замість $(t,t')$. Множину
всіх сусідів об'єкту $t$ будемо позначати
\begin{equation*}
      N_t = \{ t':tt'\in \Gamma \cup \Gamma^{-1}\}.
  \end{equation*}
Впорядковану пару $(t, k)$, $t \in T$, $k \in K$ будемо називати вершиною.
Для кожної пари сусідів $tt'\in\Gamma$ пару $((t,k),(t',k'))$, $k\in K$, $k'\in K$
будемо називати ребром. Будемо казати, що вершина  $(t^*, k^*)$ належить
розмітці $k$, якщо $k(t^*)=k^*$. Ребро $((t^*,k^*),(t^{**},k^{**}))$ належить
розмітці $k$, якщо обидві вершини $(t^*,k^*)$ та $(t^{**},k^{**})$ належать $k$.

Введемо цілочисельну функцію $q:T\times K\rightarrow\mathbb{Z}$ якостей вершин,
цілочисельну функцію $g:\Gamma\times K^2\rightarrow\mathbb{Z}$ якостей ребер. Будемо
позначати якості наступним чином: $q_t(k)$ --- якість вершини $(t,k)$,
$g_{tt'}(k,k')$ --- якість ребра $((t,k)),(t',k')$. Якістю розмітки будемо називати
функцію $G:K^T\rightarrow \mathbb{Z}$
\begin{equation*}
  G(k)=\sum_{t \in T} q_t(k_t) + \sum_{tt' \in \Gamma} g_{tt'}(k_t,k_{t'}).
  \end{equation*}
Якість розмітки --- це сума якостей всіх вершин та ребер, які їй належать.

Задача відшукання найкращої строгої розмітки полягає в тому, щоб знайти
розмітку $k^*$
\begin{equation}
      k^* \in \argmax_{k\in K^T} G(k).
      \label{eqn:strict_labeling}
  \end{equation}
Складність задачі полягає в тому, що
вона є задачею дискретної оптимізації, в якій для кожного об'єкта необхідно
вибрати лише одну мітку. Послабимо цю умову і для кожного об'єкта будемо
обирати ``суміш'' міток.

Для кожних $t\in T$ і $k\in K$ введемо $\alpha_t(k) \in \mathbb{R}$, яке ми будемо називати
вагою вершини $(t,k)$,
а для кожних $tt'\in\Gamma$, $k\in K$ і $k'\in K$ введемо $\beta_{tt'}(k,k') \in \mathbb{R}$,
яке будемо називати вагою ребра $((t,k),(t',k'))$. Позначимо $\alpha$ ---
набір $(\alpha_t(k)|t\in T, k\in K)$ ваг вершин,
$\beta$ --- набір $(\beta_{tt'}(k,k')|tt'\in\Gamma, k\in K, k'\in K)$ ваг ребер.
Пару $(\alpha, \beta)$ будемо називати ваговою функцією. Вагова функція називається
нечіткою розміткою, якщо вона задовольняє
\begin{equation*}
  \begin{cases}
    \alpha_t(k)=\sum\limits_{k' \in K} \beta_{tt'}(k,k')  , & t\in T, k\in K, t'\in N_t,\\
    \sum\limits_{k \in K} \alpha_t(k)=1, & t\in T,\\
    \beta_{tt'}(k,k')\geq 0, & tt'\in \Gamma, k\in K, k'\in K.
  \end{cases}
\end{equation*}
Якість нечіткої розмітки
\begin{equation*}
  G(\alpha, \beta)=\sum\limits_{t \in T}\sum\limits_{k\in K} \alpha_t(k)\cdot q_t(k) +
   \sum\limits_{tt'\in \Gamma}\sum\limits_{k\in K}\sum\limits_{k'\in K} \beta_{tt'}(k,k')\cdot g_{tt'}(k,k').
\end{equation*}
Задача нечіткої розмітки полягає у тому, щоб знайти розмітку з найкращою якістю
\begin{equation}
  (\alpha^*, \beta^*) = \argmax_{(\alpha, \beta)} G(\alpha, \beta).
  \label{eqn:best_rel_labeling}
\end{equation}
Неважко побачити, що, якщо обмежити значення $\alpha_t(k)$ та $\beta_{tt'}(k,k')$
лише цілими числами, то задача (\ref{eqn:best_rel_labeling}) стає еквівалентною
задачі пошуку строгої розмітки (\ref{eqn:strict_labeling}). В цьому випадку
$\alpha_t(k')=1$ в задачі нечіткої розмітки означає, що $k(t)=k'$ в задачі строгої розмітки.
