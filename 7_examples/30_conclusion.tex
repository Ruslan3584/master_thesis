\chapterConclusion

Було розглянуто узагальнений метод розв’язування $(\max ,+)$ задач розмітки із 
цілочисельними якостями на основі субградієнтного спуску. Показано, що для 
супермодулярних задач (з відомою або невідомою впорядкованістю міток) алгоритм
за скінченний час обов'язково знайде $\varepsilon$-узгоджену розмітку.
Якщо задача не є супермодулярною, алгоритм може повернути або $\varepsilon$-узгоджену
розмітку або відповідь ,,задача не є супермодулярною'' 
і відповідь обов'язково буде правильною. Також
надано спосіб перевірки отриманого розв'язку.   