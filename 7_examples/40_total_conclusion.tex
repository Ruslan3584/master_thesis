\chapter*{Висновки}
\addcontentsline{toc}{chapter}{Висновки}

Розглянуті $(\max,+)$ задачі розмітки є складними задачами дискретної оптимізації.
Для часткових випадків алгоритми точного розв'язку є занадто складними
обчислювально або вимагають виконання строгих умов для задачі.

Створено алгоритм, який розв'язує задачі класу
$(\max,+)$ та є швидшим за існуючі алгоритми.
Алгоритм відрізняється від більшості відомих ітеративних алгоритмів тим,
що надає або коректний розв'язок $(\max, +)$ задачі розмітки з цілочисельними вагами
за скінченний час, або відповідає ``задача не є супермодулярною'',
причому ця відповідь гарантовано буде вірною.
Від більшості відомих алгоритмів точного розв'язку
наведений алгоритм відрізняється тим,
що може розв'язувати не тільки супермодулярні задачі,
а також не потребує відомої впорядкованості міток.
Важливою відмінністю від роботи \cite{diffusion_shlezinger}
є зменшення максимальної необхідної кількості
запусків процедури оптимізації з $|T|\cdot|K|$ до $|T|\cdot \log_2 |K| + 1$
за допомогою використання бінарного пошуку мітки у кожному об'єкті графу.
