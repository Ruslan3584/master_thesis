\chapter*{Висновки}
\addcontentsline{toc}{chapter}{Висновки}

Розглянуті $(\max,+)$ задачі розмітки є складними задачами дискретної оптимізації.
Для часткових випадків алгоритми точного розв'язку є занадто складними
обчислювально або вимагають виконання строгих умов для задачі.

Дана робота містить постановку $(\max,+)$
задачі розмітки, огляд основних методів розв'язання. Було представлено
новий спосіб розв'язання такого класу задач, заснований на методі субградієнтного спуску.
Показано, що при достатньо нестрогих початкових умовах (супермодулярність та цілі значення якостей)
алгоритм дозволяє знайти $\varepsilon$-допустиму розмітку за скінченний час. Наведено теоретичне
обґрунтування алгоритму. Також було показано спосіб прискорення алгоритму за рахунок 
використання двійкового пошуку. Дане покращення робить цей алгоритм швидшим за багато
існуючих ітеративних методів розв’язку (водночас даючи гарантії на ,,правильність'' відповіді)
і, на відміну від алгоритмів точного розв'язку, покриває набагато ширший клас $(\max,+)$ задач розмітки.