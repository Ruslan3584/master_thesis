\chapterConclusion

Розглянуто основні етапи пошуку оптимальної розмітки.
Для оптимізації цільової функції обрано метод субградієнтного спуску, основним недоліком
якого є невизначена умова зупинки. Тому показано спосіб, який дозволяє подолати це обмеження для супермодулярних задач.
Для перевірки допустимості розмітки обрано алгоритм викреслювання другого порядку. Алгоритм було адаптовано
для використання до $(\max,+)$ задачі розмітки. Для пошуку узгодженої розмітки використано самоконтроль. Його перевагою є те, 
що не користувач вирішує, на яке питання треба відповісти, а сам алгоритм вирішує, що потрапляє у
зону його компетентності.