\chapterConclusion

В розділі було розглянуто методи отримання оптимальної розмітки за допомогою
субградієнтного методу, а також подальше відновлення найкращої розмітки після 
здійснення оптимізації цільової функції. Було показано, що при певних умовах, 
метод субградієнтного спуску збігається за скінченну кількість кроків до прийнятного 
відхилення від глобального мінімуму. Також розглянуто алгоритм викреслювання другого порядку, 
який дозволяє для репараметризованих якостей задачі перевіряти, чи існує допустима розмітка.
В свою чергу, самоконтроль дозволяє знайти допустиму розмітку, якщо вона існує, або ж
просигналізує про те, що якась із початкових умов задачі не виконується.
