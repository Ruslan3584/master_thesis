\chapterConclusion

Розроблено алгоритм розв’язування $(\max ,+)$ задач розмітки із
цілочисельними якостями на основі субградієнтного спуску.
Наведений алгоритм потребує не більше, ніж $|T|\cdot \log_2 |K| + 1$
запусків субградієнтного спуску,
на відміну від відомого алгоритму \cite{diffusion_shlezinger},
для якого може знадобитись $|T|\cdot|K|$ викликів алгоритму дифузії.
Зауважимо, що замість субградієнтного спуску можна використовувати
будь-який інший алгоритм, що гарантує скінченний час роботи
для будь-якого наперед заданого додатного рівня похибки.
Показано, що для
супермодулярних задач (з відомою або невідомою впорядкованістю міток) алгоритм
за скінченний час обов'язково знайде $\varepsilon$-допустиму розмітку.
Якщо задача не є супермодулярною, алгоритм може повернути або $\varepsilon$-допустиму
розмітку або відповідь ``задача не є супермодулярною''
і відповідь обов'язково буде правильною.
Наведено спосіб перевірки отриманого розв'язку.
