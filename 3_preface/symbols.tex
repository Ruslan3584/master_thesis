\chapter*{Перелік умовних позначень, символів, одиниць, скорочень і термінів}
\addcontentsline{toc}{chapter}{Перелік умовних позначень, символів, одиниць, скорочень і термінів}

\textbf{Стандартні позначення}

 \noindent$\mathbb{N}$~---~множина натуральних чисел, \\
 \noindent$\mathbb{Z}$~---~множина цілих чисел, \\
 \noindent$\mathbb{R}$~---~множина дійсних чисел, \\
 \noindent$\mathbb{N}_0$~---~множина $\mathbb{N}$ з $0$ ($\mathbb{N}_0=\mathbb{N}\cup\{0\}$), \\
 \noindent$\mathbb{R}_+$~---~множина невід'ємних чисел з $\mathbb{R}$, \\
 \noindent$\emptyset $~---~порожня  множина, \\
 \noindent$X^n$~---~$n$-вимірний векторний простір над множиною $X$, \\
 \noindent$\langle x, y\rangle $~---~скалярний добуток векторів $x,y$, \\
 \noindent$\left| X \right| $~---~потужність множини $X$, або кардинальне число множини $X$, \\
 \noindent$X \times Y$~---~декартів добуток множин $X$ та $Y$, \\
 \noindent$X \cup Y$~---~об'єднання множин $X$ та $Y$. \\

\textbf{Позначення, введені в дисертації}

 \noindent$T$~---~множина об'єктів, \\
 \noindent$K$~---~множина міток, \\
 \noindent$\Gamma$~---~структура сусідства, \\
 \noindent$N_t$~---~множина усіх сусідів об'єкту $t$. \\
