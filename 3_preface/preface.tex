\chapter*{Вступ}
\addcontentsline{toc}{chapter}{Вступ}

Роботу виконано у межах теми ``Створення інтелектуальних інформаційних
технологій на базі методів і засобів образного мислення'',
державний реєстраційний номер 01114U002068.

\textbf{Актуальність роботи.}
Задачі розмітки відіграють важливу роль у структурному розпізнаванні
зображень. Одним з важливих класів задач
розмітки, які мають ефективний розв’язок, є клас супермодулярних
\((\max,+)\) задач з відомою  та невідомою впорядкованістю міток.
Ми наводимо алгоритм розв'язку задач, який гарантовано за скінченну кількість
кроків видасть правильну відповідь, а також є швидшим за існуючі методи.

\textbf{Мета і завдання дослідження.}

\textit{Об'єкт дослідження}~---~\((\max,+)\) задачі розмітки.

\textit{Предмет дослідження}~---~точний розв’язок задач розмітки.

Метою роботи є створення алгоритму, який розв'язує задачі класу
\((\max,+)\) та є швидшим за існуючі алгоритми.

Завдання наступні:
\begin{enumerate}
  \item
    проаналізувати існуючі алгоритми розв'язку \((\max,+)\) задач розмітки,
  \item
    розробити алгоритм розв'язку \((\max,+)\) задач розмітки, який буде
    швидшим за існуючі методи з гарантією збіжності за скінченну
    кількість кроків,
  \item
    реализувати алгоритм програмно та провести експерименти.
\end{enumerate}

\textbf{Наукова новизна одержаних результатів.}

Створено алгоритм розв'язку \((\max,+)\) задач розмітки, який швидший за
існуючі аналоги та має гарантію збіжності за скінченну кількість кроків.

\textbf{Практичне значення одержаних результатів.}

За допомогою алгоритму можна отримати розв'язок для будь-яких
супермодулярних і деяких не супермодулярних \((\max,+)\) задач розмітки
за скінченну кількість кроків,
а для інших задач отримати відмову від розпізнавання.

\textbf{Публікації.}

Стаття ``Алгоритм розв'язку супермодулярних $(\max, +)$ задач
розмітки з самоконтролем на базі субградієнтного спуску'' (Кригін В. М., Хоменко Р.О.)
пройшла повний цикл рецензування, доопрацювання та схвалена для
публікації у журналі
,,Кібернетика та системний аналіз'' у № 4, 2022 р.

Автор висловлює вдячність за корисні поради в написанні роботи Кригіну Валерію Михайловичу~---~молодшому
науковому співробітнику Відділу обробки та розпізнавання
образів Міжнародного науково-навчального центру інформаційних
технологій і систем НАН України та МОН України.